\documentclass{article}
\usepackage[utf8]{inputenc}
\usepackage{amsmath}
\usepackage[spanish]{babel}
\title{Apuntes de programación lineal}

\author{Lanolyn Rodriguez}
\begin{document}
\maketitle
\tableofcontents
\section{Forma ESTANDAR}
\label{sec:informacion-que-cura}

La forma estandar de un problema de programación lineal es: Dados una
matriz $A$ y vectores $b,c$, maximizar $c^Tx$ sujeto a $Ax\leq b$.



\section{Forma SIMPLEX}
\label{sec:informacion-que-cura}

La forma simplex de un problema de programación lineal es: Obteniendo un
problema en forma estandar con una matrizmatriz $A$ y vectores $b,c$, a
maximizar $c^Tx$ sujeto a $Ax\leq b$ sacamos una variables
$Z_1,Z_2,...,Z_i$ si son i ecuaciones sujetas a la funcion $c^tx$ a maximizar, que se
igualan a las funciones sujetas $z_i=$


\begin{tabular}{|c|c|c|}
  \hline
  & A & A \\
  \hline
  Maquina1 & 1 & 2 \\
  Maquina2 & 1 & 1 \\
  \hline               
\end{tabular}

\begin{equation}
  A=
  \label{eq:1}
  \begin{pmatrix}
    0&i&2\\
    3&-1&5
  \end{pmatrix}
\begin{pmatrix}
  0&i&2\\
  3&-1&5
  \end{pmatrix}
\end{equation}


\end{document}
